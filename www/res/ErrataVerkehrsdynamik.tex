%Siehe auch Errata.txt

\documentclass[11pt,a4paper]{scrartcl}
\usepackage[ngerman]{babel}
\usepackage{units,booktabs}
\usepackage[dvips]{hyperref}
\usepackage{graphicx}
\usepackage{amsmath, amssymb}

%\include{LokaleDefs}  %% eigene command definitionen (AK)
\providecommand{\gquote}[1]{\glqq #1\grqq}  %german,left/right,2xquote
\providecommand{\abl}[2]   {\frac{\D #1}{\D #2}}  % =\abltot
\providecommand{\ablpart}[2]{\frac{\partial #1}{\partial #2}}  %d(#1)/d(#2)
\providecommand{\hats}{\hat{s}}
\renewcommand{\abl}[2]{\frac{{\rm d} #1}{{\rm d} #2}}  % =\abltot
\providecommand{\bdm}{\begin{displaymath}}
\providecommand{\edm}{\end{displaymath}}
\providecommand{\3}{{\ss}}
\providecommand{\gquote}[1]{\glqq #1\grqq}  %german,left/right,2xquote
%% Koma-Headings
\usepackage[komastyle,automark,autooneside,headsepline]{scrpage2}
\pagestyle{scrheadings}
\setkomafont{pagehead}{\normalfont} %\sffamily}  %\bfseries
\setkomafont{pagenumber}{\normalfont\rmfamily} %%\slshape}
\markboth{Errata zum Lehrbuch \gquote{Verkehrsdynamik und -simulation}}{Errata zum Lehrbuch \gquote{Verkehrsdynamik und -simulation}}

\title{Errata zum Lehrbuch\\\gquote{Verkehrsdynamik und \mbox{-simulation}}}
\author{Martin Treiber und Arne Kesting}
\date{Oktober 2012}
\begin{document}
\maketitle
\emph{Hinweis:}
In der folgenden Liste  werden nur inhaltliche, nicht aber rein
sprachliche Fehler aufgelistet.

\begin{itemize}

\item
\textbf{Kap. 6.1, Seite 51:}
In den 1990er, nicht 1930er Jahren f\"uhrte eine gesteigerte
Rechenleistung ... zu verst\"arkten Aktivit\"aten [auf dem Gebiet der
  Verkehrsdynamik].

\item
\textbf{Kap. 8.4, Seite 92, Gleichung~(8.21):}
In dieser Formel muss $\rho_\text{max}$ durch $Q_\text{max}$ ersetzt
werden.

\item
\textbf{Kap. 8.5, Seite 110:}
In Gleichung~(8.45) wurde nach dem ersten Gleichheits\-zeichen die Verteilungsfunktion $F_N$  mit der Dichtefunktion $f_N$ der Normalverteilung vertauscht. Die Gleichung lautet 
\begin{equation*}
g(x,t)=f_N^{(\mu, \sigma^2)}(x)
 = \frac{1}{\sqrt{4\pi D t}} \exp\left[- \frac{(x-\tilde{c}t)^2}{4Dt} \right].
\end{equation*}

\item
\textbf{Kap. 9.4.2, Seite~125:}
Der Verkehrsdruck ist nicht durch $\theta_0$ sondern durch
$\rho\theta_0$ gegeben [drei Zeilen unter Formel~(9.17)].

\item
\textbf{Kap. 9.4.3, GKT-Formel~(9.22):}
Das Argument des Boltzmannfaktors ist falsch: Es gilt
\begin{equation*}
B\left(\frac{V-V_a}{\sqrt{2} \sigma_V}\right)
\end{equation*}
anstelle von
\begin{equation*}
B\left(\frac{V-V_a}{\sigma_V}\right).
\end{equation*}

\item
\textbf{Kap. 10.5, Abb.~10.3:}
Die verwendete OV-Funktion ist durch Gleichung~(10.18), nicht durch~(10.19)
gegeben.
 
\item
\textbf{Kap. 10.5, Formel~(10.20):}
Auf de rechten Seite wurden $v_{\alpha}(t)$ und $v_{\alpha-1}(t)$ vertauscht.
 
\item
\textbf{Kap. 10.6, Abb.~10.5:}
Die verwendete OV-Funktion ist ebenfalls durch Gleichung~(10.18), nicht durch~(10.19)
gegeben.

\item
\textbf{Kap. 11.1, Seite~155 zwischen den Gleichungen~(11.2) und~(11.3):}
\gquote{je schneller dieses f\"ahrt} $\Rightarrow$ \gquote{je langsamer dieses
f\"ahrt}

\item
\textbf{Kap. 12.3, Seite~177 zwischen den Gleichungen~(12.5) und~(12.6):}
Die Time-to-Collision ist inkorrekt definiert. In der Formel im Text
muss es lauten $\tau_\text{TTC}=s/\Delta v$, nicht $\tau_\text{TTC}=\Delta
v/s$. Die Formel (12.5) ist korrekt.

\item
\textbf{Kap. 12.5, S. 181 Mitte, Formel (12.16):}
In der Definition der summierten L\"ucke $s_{\alpha\beta}$ geht die
Summe von $j=0$ bis $\alpha-\beta-1$, nicht bis $\beta-1$:
\bdm
s_{\alpha\beta}=\sum\limits_{j=0}^{\alpha-\beta-1}s_{\alpha-j}
\edm

\item
\textbf{Kap. 12.5, S. 182 Mitte:}
Die Ungenauigkeit bei der Sch\"atzung der relativen Ann\"aherungsrate
(inverse TTC) ist  $\unit[0.01]{s^{-1}}$, nicht  $\unit[0.01]{s}$.

\item
\textbf{Kap. 14.3.4, Seite~203 vor Formel~(14.9):}
Der Text ist etwas ungenau und ohne Ber\"ucksichtigung der Einheiten
formuliert. Der pr\"azisierte Text zwischen den Gleichungen~(14.8) 
und~(14.9) lautet
\gquote{Da sowohl $T$ als auch $\tau$ in der Gr\"o\3enordnung von \unit[1]{s} liegen
(vgl. Tabelle~10.1)
%\ref{tab:param-OVM}
und die Beschleunigungen $b_\text{safe}$
und $\Delta a$ von der Gr\"o\3enordnung $\unit[1]{m/s^2}$ bzw. kleiner
sind (Table~14.1),
%\ref{tab:MOBIL}
sind alle Beitr\"age, welche das Produkt
$\tau T$ enthalten, von
der Gr\"o\3enordnung \unit[1]{m} oder kleiner und
damit gegen\"uber den L\"ucken $s_{\alpha}$, $\hat{s}_{\alpha}$ und
$\hat{s}_\text{hz}$ vernachl\"assigbar. Im Ergebnis bekommt man die Bedingungen.}

\item
\textbf{Seite~204, Formel~(14.11):}
Das Anreizkriterium f\"ur das Full Velocity Difference Modell lautet
\bdm
\hats_{\alpha} > s_\text{e} \big(v_\text{opt}(s_{\alpha})+\tau \ \Delta a
% + \gamma\tau(v_{\hatb}-v_{\alpha}) \big) & \text{incentive}.
 + \gamma\tau(v_v-v_{vz}) \big)
\edm

\item
\textbf{Kap. 15, Seite~218, Formel~(15.11):}
Die Ableitungen der Beschleunigungsfunktion sind partielle, nicht
totale Ableitungen, also $\ablpart{a_\text{mic}}{s}$ statt
$\abl{a_\text{mic}}{s}$ usw.

\item
\textbf{Kap. 15.4, Seite~228, dritte Gleichung im Text:}
$q_1=-iV_ep_0+i\rho_eV'_e p_0$ anstelle von
$q_1=-iV_ep_0+i\rho_eV'_e$.

\item
\textbf{Kap. 15.4, Seite~232, Formel~(15.67):}
Der Reaktionszeit- bzw. Folgezeitparameter ist durch $T$ anstelle von $T_r$ gegeben. 
Ferner wurde die Geschwindigkeit des F\"uhrungsfahrzeugs irrt\"umlich mit $v_p$ anstelle von $v_l$ bezeichnet.

\item
\textbf{Kap. 15.4, Seite~232, Formel~(15.68):}
Die partielle Ableitung $a_{v_l}$ ist durch
\bdm
a_{v_l}=\frac{v_e}{T(bT+v_e)}
\edm
anstelle von $a_{v_l}=\frac{v_e}{bT+v_e}$ gegeben.

\item
\textbf{Kap. 15.5, S. 235, Formel (15.76):}
Die Formel lautet
\bdm
\tilde{U}(x, t) \propto  \text{exp} \left[ \text{i}(k_0\sup{phys} x-\omega_0 t) \right] \
\text{exp} \left[ \left(
 \sigma_0 
 -\frac{\left(v_g-\frac{x}{t}\right)^2}{2(\text{i}\omega_{kk}-\sigma_{kk})}\right) t
\right].
\edm


\item
\textbf{Kap. 17.1, Seite 261:}
Der Stau auf der A8-Ost ist in Abb. 17.1b, nicht 17.1a gezeigt.

\item
\textbf{Aufgabe 15.8, S.274:} In Teilaufgabe 2 sollen nicht $n_1(t)$ und $n_2(t)$ sondern $N_1(t)$ und $N_2(t)$ bestimmt werden.

\item
\textbf{Kap. 19.6, Seite 281:} In der zweiten Gleichung von~(19.9) fehlt ein Faktor von $2\pi$, so 
dass f\"ur den effektiven Mitteldruck bei Viertaktmotoren gilt 
\begin{equation*}
\bar{p}=\frac{4 \pi M}{V_\text{zyl}}.
\end{equation*}

\item
\textbf{S. 282, Bildunterschrift von Abb.~19.2:} \unit[11]{Liter/kWh} statt \unit[11]{kg/kWh}.

\item
\textbf{Kap. 20.5, Bildunterschrift zu Abb. 20.6, S.298:} Es handelt sich um gr\"o\3ere Einfahrten, nicht Ausfahrten.

\item
\textbf{Kap. 20.7, S. 299:} In den Gln (20.3) und (20.4) steht $a_0^2$ im Nenner, nicht $a_0$.

\item
\textbf{Kap. 20.7, S. 300:} Ersetze \gquote{die Konstante $\dot{C}$} durch \gquote{die Konstante $\dot{C}_0$}.

\item
\textbf{S. 318, L\"osung der Aufgabe 8.4} Die L\"osung ist zwar
n\"aherungsweise, aber nicht exakt korrekt. Die richtige L\"osung
lautet
\bdm
\tau_\text{tot}=\frac{1}{2}\rho_\text{max}\tau^2 \ 
\frac{c_\text{up}c_\text{cong}}{c_\text{up}-c_\text{cong}}
\edm
mit
\bdm
c_\text{up}=\frac{Q_\text{in}}{Q_\text{in}/V_0-\rho_\text{max}}, \quad
c_\text{cong}=-\frac{1}{\rho_\text{max}T}.
\edm

\item
\textbf{S. 321, L\"osung der Aufgabe 8.5, Teilaufgabe 4} Es handelt
sich um die stromabw\"artige, nicht stromaufw\"artige Staufront.

\item
\textbf{S. 346, L\"osung der Aufgabe 13.3, erste Formel}
Ersetze $V'_e(0)$ durch $Q'_e(0)$ und $V'_e(\rho_\text{max})$ durch $Q'_e(\rho_\text{max})$

\item
\textbf{S. 353 und S.354, L\"osungen der Aufgaben 15.8 und 18:2} Die Bilder zu dne beiden L\"osungen wurden vertauscht. 

\item
\textbf{S. 358, L\"osung der Aufgabe 19.6:} Der Integrand in der
ersten eckigen Klammer hat einen falschen Luftwiderstandsbeitrag: 
$\frac{1}{2} c_w \rho_L A a^3t^3$ statt
$\frac{1}{2} c_w \rho_L A a^2t^2$. (Die n\"achsten Formeln sind wieder
richtig.)

\end{itemize}

\end{document}















